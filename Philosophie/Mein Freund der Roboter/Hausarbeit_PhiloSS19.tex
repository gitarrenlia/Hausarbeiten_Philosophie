\documentclass[12pt]{article}
\usepackage[a4paper, left=3.5cm, right=3.5cm, top=3.5cm, bottom=3.5cm]{geometry}
\usepackage{setspace}
\setstretch{1,33}%20/12
\usepackage[german]{babel} 
\usepackage[utf8]{inputenc}
\usepackage{graphicx}
\usepackage{hyperref}
\usepackage{microtype}
\begin{document}

%
% ---- Titelseite ----
%
\begin{titlepage}
	\centering
	\includegraphics[width=0.5\textwidth]{logo.png}\par\vspace{1cm}
	{\scshape\LARGE Philosophisches Institut \par}
	\vspace{1cm}
	{\scshape\Large Seminar Philosophie\par}
	{\scshape\large (Roboterethik)\par}
	\vspace{1.5cm}
	{\huge\bfseries Mein Freund der Roboter\par}
	{\bfseries  \large Von der Möglichkeit einer Freundschaft zwischen Mensch und Maschine\par}
	\vspace{2cm}
	{\Large\itshape Lia Schulze Dephoff\par}
	{\itshape Studiengang Informatik Master\par}
	{\itshape Matr.Nr. 375625\par}
	\vfill
	Dozentin:\par
	\textsc{Carmen Krämer}

	\vfill

% Bottom of the page
	{\large \today\par}
\end{titlepage}

%
% ---- Inhaltsverzeichnis ----
%
\newpage
\tableofcontents 
\newpage
%
% ---- Einleitung ----
%
\section{Einleitung}
Das Thema Freundschaft scheint so alt zu sein wie die Menschen selbst. Schon im Kindesalter beginnen die Menschen damit sich Freunde zu suchen, um mit ihnen zu spielen und nicht alleine sein zu müssen. Später im Leben haben Freundschaften dann zwar eine etwas andere Bedeutung oder einen anderen Zweck, aber der Wunsch nach Freundschaften bleibt bestehen. Wie schon Aristoteles sagte: \glqq Ohne Freundschaft möchte niemand leben\grqq (Aristoteles, Buch 8, Kapitel 1) und Cicero erklärte wieso: \glqq[Freundschaft lässt] unser Glück in schönem Lichte erscheinen; andererseits erleichtert sie Unglück und Mißgeschick durch Teilnahme und Mitgefühl [...][und sie] erfüllt uns auch für die Zukunft mit freudiger Hoffnung und lässt den Muth nicht sinken\grqq (Cicero, s.44).\newline
Freundschaft gibt den Menschen also viel Freude in guten Zeiten und bietet einen Halt an weniger guten Tagen.\newline
In einer Welt, in der die Technik sich immer mehr verbreitet und viele Bereiche unseres Lebens immer mehr automatisiert und technisiert werden, scheint es kaum noch Grenzen zu geben. So könnte man sich auch im Bereich der Freundschaft fragen, ob sich diese mit Technik ersetzen ließe. \newline
Im Film wird bereits mit solchen Mensch-Roboter-Freundschaften gespielt. So erleben Hiro und Baymax im Film \glqq Baymax\grqq{} gemeinsam ein Abenteuer und der Zuschauer bekommt den Eindruck, dass beide mit der Zeit enge Freunde werden. Auch in Robot \& Frank entwickelt sich aus einer zunächst ablehnenden Haltung des Senioren seinem neuen Pflegeroboter gegenüber eine immer engere Freundschaft.\newline
Aber könnte so ein Szenario auch in der Realität funktionieren? Können Menschen und Roboter Freunde sein? Und wenn ja, wie müsste so ein Roboter aussehen und funktionieren, um Teil einer Freundschaft zu sein? Um diese Fragen zu beantworten, wird in der vorliegenden Hausarbeit zunächst analysiert, wie ein solcher Roboter aussehen sollte, um beim Menschen möglichst viel Vertrauen zu wecken und wie er sich zu verhalten hätte.\newline
Im Hauptteil geht es dann um den Begriff der Freundschaft nach Aristoteles und Cicero und ob sich dieser mit einem Roboter vereinbaren lassen könnte.\newline
Schließlich wird ein Roboterfreund noch anhand von Beispielen in der Pflege und Kindererziehung diskutiert, bevor ein zusammenfassendes Schlusswort angebracht wird.

\section{Zusammenhang Vertrauen und Anthropomorphismus}
Studien zeigen, dass Menschen von der Erscheinungsweise eines Roboters auf seine Eigenschaften schließen (vgl. Jens Koolwaay 2018, s.26). Hat ein Roboter also eine anthropomorphe Erscheinungsweise, verhalten sich Menschen ihm gegenüber ähnlich wie mit einem realen Menschen und erwarten das gleiche von ihm.\newline
Anthropomorphismus ist ein Kompositum der griechischen Worte anthropos, der Mensch und morphe, die Gestalt. Es bezeichnet das Übertragen von menschlichen Eigenschaften auf Nichtmenschliches. Bei Robotern wird häufiger das Adjektiv humanoid verwendet, wenn es um Roboter mit menschlicher Gestalt geht. Allerdings reicht ein menschliches Aussehen nicht vollständig aus, um beim Menschen Vertrauen zu wecken. \glqq Menschen zeigen jedoch besonders dann großes Vertrauen zu einem Roboter, wenn dessen Grad an Anthropomorphismus sich mit der Seriosität seiner Tätigkeit deckt\grqq(Jens Koolwaay 2018, s. 27). Wenn einem also ein humanoider Roboter gegenüber steht, rechnet man damit, dass er sich verhält wie ein Mensch. Tut er dies dann nicht, können wir nicht mehr einschätzen, wie er sich verhält und er wirkt auf uns eher gruselig und keineswegs vertrauenswürdig. Dieses Verhältnis zwischen Anthropomorphismus und Vertrautheit würde man daher zunächst als stetig steigende Kurve vermuten. Dies ist allerdings nicht so. Der japanische Robotiker Masahiro Mori bezeichnet die tatsächliche Kurve als \glqq Uncanny-Valley-Effekt\grqq{} oder auf Deutsch etwa der Effekt des unheimlichen Tals. Die Kurve hat an einer Stelle einen deutlichen Einbruch der Vertrautheit, bevor sie bei weiter steigender Vermenschlichung schließlich wieder ansteigt. (Siehe Abbildung\ref{img:uncanny-valley})
\begin{figure}
\centering
\includegraphics[scale=0.35]{Mori_Uncanny_Valley_de.png}
\caption{Uncanny-Valley-Effekt}
\label{img:uncanny-valley}
\end{figure}
Es wird vermutet, dass dieser Effekt eine Reaktion auf die Inkongruenz zwischen menschlichem Aussehen und Verhalten darstellt. Demnach seien Roboter, welche dem Menschen in der Gestalt besonders ähneln, aber in ihren Handlungen teilweise vom Menschen abweichen, besonders anfällig für den Effekt. Langsame und stockende Bewegung könnten so zum Beispiel für weniger Vertrauen sorgen (vgl. Birte Schiffhauer 2015, s.11-12).\newline
Um die Gefahr zu vermeiden, dass ein humanoider Roboter im dunklen Tal eingeordnet wird, werden oft bevorzugt kleine, niedliche, puppenähnliche Roboter eingesetzt (Oliver Bendel, Pflegeroboter, 2018). Diesen werden dann durch ihr Erscheinungsbild weniger seriöse Tätigkeiten zugetraut und dadurch kann schließlich ein vertrauteres Verhältnis zum Roboter aufgebaut werden.\newline
Doch auch nicht-humanoide Roboter können im Menschen Gefühle auslösen. Der Staubsauger Roboter Roomba scheint bei seinen Besitzern tiefe emotionale Dankbarkeit hervorzurufen. So gaben die Konsumenten an, dass sie für den Roomba putzen würden, damit er sich ausruhen könne, ihn mit in den Urlaub nehmen und ihm einen Namen geben (vgl. Matthias Scheutz 2009).\newline
Es scheint also so zu sein, dass Menschen grundsätzlich in der Lage sind Robotern gegenüber eine emotionale Bindung aufzubauen. Sie müssen dazu scheinbar nicht humanoid sein, aber falls sie es sind, sollte ihre Erscheinung mit ihrem Verhalten kohärent sein. \glqq Weizenbaum (1976) wiederum beschreibt Anthropomorphisierung als psychologische Konsequenz einer emotionalen Bindung an eine Maschine\grqq  (Birte Schiffhauer, s. 10). So versuchen die Menschen also vielleicht nur deshalb die Roboter humanoid zu gestalten, weil sie diese emotionale Bindung bereits aufgebaut haben und sich nun eine für diese Emotionen gewohnte Gestalt für sie wünschen. Dabei bleibt die Frage offen, ob ein nicht humanoider Roboter vielleicht viel mehr Vertrauen bei den Menschen wecken würde und vielleicht sogar besser geeignet wäre als Freund der Menschen.

 \section{Soziales Verhalten von Robotern}
 
Wie im letzten Kapitel zu erkennen ist, hängt die Beziehung des Menschen zum Roboter nicht nur von seiner Erscheinung ab, sondern auch von seinem Verhalten. Er soll sich passend zu seinem Aussehen verhalten. Doch das ist noch nicht genug. Sein Verhalten muss außerdem dazu führen, dass der Mensch ihn \glqq als eigenständige Persönlichkeit mit konsistenten Verhaltensweisen\grqq (Oliver Bendel 2018, s. 66) akzeptiert. Gerade so ein konsistentes Verhalten ist besonders wichtig, um Vertrauen aufzubauen. Denn je konsistenter das Verhalten ist, desto besser kann der Mensch einschätzen, wie sich die Maschine verhält und umso mehr kann er ihr vertrauen. Für eine eigene Persönlichkeit ist aber auch das emotionale Verhalten relevant. Ein Roboter müsste Emotionen zeigen und verstehen und damit umgehen können.\newline
Unter der Annahme, dass ein Roboter keine echten Emotionen haben kann, muss es also trotzdem gelingen es so scheinen zu lassen, dass wir glauben der Roboter empfinde Empathie und verstünde wie wir denken. Die Soziologin Sherry Turkle sieht jedoch genau darin eine Gefahr. Für sie sei es schwierig Vertrautheit aufzubauen zu einer Maschine, die nur so tut, als hätte sie Gefühle. Für sie spielt die Authentizität eine wichtige Rolle und diese entstünde aus einem gemeinsamen Pool von menschlichen Erfahrungen, welche ein Roboter nicht haben kann (vgl. Sherry Turkle 2017).\newline
Leben und Tod, sowie Familie sind tatsächlich ein Faktor, der Menschen tagtäglich beschäftigt und ein Roboter kann offensichtlich weder sterben, noch hat er eine Familie. Wenn aber ein Roboter uns vormachen kann, dass er beispielsweise emotionale Zuneigung zu uns empfindet, wieso sollte er uns dann nicht auch den Eindruck vermitteln können er habe Angst vor dem Tod oder frage sich, wofür wir leben? Den Tod könnte man letztlich gut vergleichen mit dem Defekt eines Roboters, der nicht mehr repariert werden kann oder zum Beispiel aus Kosten Gründen nicht mehr repariert werden soll. \newline
Wenn es allerdings um Erinnerungen geht könnte es dabei schon anders aussehen. Zwar könnte ein Roboter so implementiert werden, dass er sich an Situationen erinnert und diese auch wieder nacherzählen kann, allerdings hat er andere Erfahrungen der Vergangenheit als wir Menschen. Ihm eine fiktive Kindheit einzuprogrammieren könnte eine Lösung für das Problem sein. Jedoch fehlt es dann wieder an Individualität da so ein Roboter vermutlich nicht als Unikat verkauft werden würde. Außerdem wäre so eine einprogrammierte Kindheit sehr unglaubwürdig, da der Mensch schließlich immer noch weiß, dass es sich um eine Maschine und keinen echten Menschen handelt. Der Roboter würde also deutlich an Authentizität verlieren. Dabei kommt es allerdings auch wieder auf den individuellen Nutzer an. Kindern oder psychisch kranken Menschen könnte man deutlich mehr vorspielen, als gesunden Erwachsenen. Doch hier stellt sich dann wieder die moralische Frage, wie weit man die Situation von benachteiligten Personen für solche Zwecke ausnutzen darf. Bietet es für sie eher einen Vorteil oder greift es ihre Würde zu sehr an? Dies ist eine schwierige moralische Frage, auf die ich hier nicht weiter eingehen werde.\newline
Am Ende könnte der Roboter uns wahrscheinlich alles Mögliche vormachen, doch besteht dann immer mehr die Gefahr, dass er nicht mehr als authentische Persönlichkeit wahrgenommen wird.\newline
Ein weiterer schwieriger Aspekt könnten politische Meinungen sein. Auf der einen Seite stellt sich die Frage, ob man einer Maschine überhaupt politische Meinungen einprogrammieren darf und damit potenziell deren Besitzer beeinflussen könnte und auf der anderen Seite, wie sich eine konträre politische Meinung auf das Verhältnis zwischen Mensch und Maschine auswirkt. Laut Cicero ist eine Verschiedenheit der politischen Meinungen eine Bedrohung für die Freundschaft (vgl. Cicero, s.44). Dies kann sicherlich zutreffen. Wenn beide Diskutierenden an ihrer Meinung festhalten und sich nicht überzeugen lassen, kann dies schnell zu einem Streit führen oder dem Wunsch das Gespräch so schnell es geht zu beenden. Meinungsimplementation ist also ein großer und nicht trivialer Faktor, wenn es um die Persönlichkeitsprogrammierung von sozialen Robotern geht.\newline
 \newline
 
%
% ---- Hauptteil ----
%
\section{Was ist Freundschaft und kann ein Roboter Teil davon sein?}
Um der Definition von Freundschaft näher zu kommen, betrachte ich zunächst die Definition des Dudens: \glqq auf gegenseitiger Zuneigung beruhendes Verhältnis von Menschen zueinander \grqq (Duden). Das würde also bedeuten, dass eine Freundschaft nur dann entstehen kann, wenn zwei Menschen sich gegenseitig mögen. Auf der einen Seite steckt also schon in der Definition, dass Freundschaft nur zwischen Menschen bestehen kann und auf der anderen der Punkt, dass beide Seiten Zuneigung füreinander empfinden müssen. Auch Aristoteles schließt sich dieser Definition an, indem er schreibt: \glqq Nun spricht man aber bei der Liebe zu leblosen Dingen nicht von Freundschaft. Denn hier ist keine Gegenliebe noch Wohlwollen vorhanden\grqq (Aristoteles, Buch 8, Kapitel 2).\newline
Aristoteles schließt mit diesem Satz allerdings auch keine Tiere aus. Vielleicht wäre es also bei Tieren möglich mit ihnen befreundet zu sein, wenn gegenseitiges Wohlwollen vorhanden ist. Aber auch einem Roboter könnte man schon eine Art von Lebendigkeit zuordnen, zumindest auf einer fiktiven Ebene. Es scheint, als würde der Roboter leben, weil er menschlich handelt und uns den Eindruck vermittelt, er sei wie ein Mensch. Wenn Tiere also für eine Freundschaft geeignet wären nur, weil sie leben, könnte es vielleicht auch ein Roboter sein, wenn er die Aspekte für eine Lebendigkeit glaubhaft genug verkörpert.\newline
Schließlich sagt man auch, dass der Hund des Menschen bester Freund sei. Wenn es also ein Hund kann, dann liegt vielleicht auch der Roboter gar nicht mehr in so weiter Ferne.\newline
Schwieriger ist wohl eher der Punkt der Liebe und des Wohlwollens. Wir haben bereits die Annahme getroffen, dass Maschinen keine echten Emotionen empfinden können, sondern diese höchstens sehr realitätsnah vorspielen. Demnach wäre eine Freundschaft zu einem Roboter wohl eher eine fiktive oder vorgespielte Freundschaft und keine reale. Allerdings kann auch eine vorgespielte Freundschaft unter Menschen existieren, indem ein Mensch den anderen zu seinem Vorteil ausnutzt. Dabei spielt der angebliche Freund dem Gegenüber vor ihn zu mögen, um dafür eine Gegenleistung zu erhalten. Würde diese Gegenleistung nicht mehr funktionieren, wäre auch die Freundschaft nicht mehr existent, da sie keinen Nutzen mehr hätte. Diese Art der Freundschaft ist wohl von außen betrachtet nicht als eine wahre Freundschaft anzusehen, für den Ausgenutzten könnte es allerdings zunächst so scheinen. \newline

\subsection{Freundschaft unter den Gleichen}
Cicero ergänzt in seiner Definition von Freundschaft: \glqq Denn sie setzt die vollkommenste Uebereinstimmung in allen göttlichen und menschlichen Dingen [...] voraus\grqq (Cicero, s.44). Freundschaft könne also nur dann entstehen, wenn beide Seiten sich in jeglichen Dingen gleichen. Diese Dinge spezifiziert er als er über sein Verhältnis zu seinem Freund Scipio berichtet. \glqq Scipio [...] mit dem ich [...] die vollkommenste Uebereinstimmung der Vorsätze, Bestrebungen und Grundsätze hatte\grqq(Cicero, s.53). Freundschaft basiert danach, also unter Anderem auf gleichen Interessen, gleichen Ansichten und gleicher Wesensart.
So konnte eine Studie der Universität Bielefeld in 2011 zeigen, dass Menschen einen Roboter, der scheinbar eher dem eigenen Kulturkreis angehört positiver wahrnehmen, als einen Roboter aus einem anderen Kulturkreis. In der Studie wurde für die befragten deutschen Studierenden nur der Name des Roboters von Armin zu Arman getauscht und vorgegeben, dass Armin von einer deutschen Universität und Arman von einer türkischen Universität entwickelt worden sei. Dieser simple Trick reichte aus, um die Ergebnisse für den Roboter aus der eigenen Gruppe besser ausfallen zu lassen, als jene für den der ausländischen Gruppe. Bewertet wurden dabei die Attribute Design, persönliche Wärme, Verstand, psychologische Nähe und ob die Befragten mit dem Roboter zusammen leben wollen würden oder zumindest gerne mit ihm sprechen wollen (vgl. Eyssel und Kuchenbrandt 2011, s.726-728). Die bewerteten Punkte könnten alle verwendet werden, um herauszufinden, ob jemand mit diesem Roboter befreundet sein möchte. Die Fragen über den Verstand waren so formuliert, dass aus ihnen deutlich wurde,  wie sehr die Studierenden den Roboter vermenschlichen. Wie aber in den vorigen Kapiteln festgestellt wurde, reicht eine Vermenschlichung nicht aus, um Vertrauen aufzubauen. Dieses Vertrauen konnte aber durch die Attribute persönliche Wärme und psychologische Nähe besser herausgefunden werden. Besonders aber die Frage danach, ob man mit dem Roboter reden oder sogar zusammen leben wollen würde, zeigt, wie weit man in dieser Maschine vertrauen würde und offenbar ist so ein Vertrauen zu Robotern, die einem selbst ähneln höher als zu Robotern aus fremden Kulturkreisen.\newline
Doch ähnliche kulturelle Herkunft ist auch nicht alles was zur Gleichheit gehört. Gleiche Interessen und gleiche Ansichten können zwar in vielen Fällen aus der Kultur entstehen, aber auch innerhalb einer Kultur gibt es Menschen mit unterschiedlichen Interessen und Meinungen. Cicero hält diese beiden Dinge für grundlegend wichtig und auf den ersten Blick erscheint auch relativ klar zu sein wieso. Interessen sind häufig die Grundlage für Gespräche und unter Freunden werden sicherlich einige Gespräche geführt. Wenn nun die Interessen auf beiden Seiten gleich sind, führt das natürlich zu spannenderen Gesprächen aus der Sicht beider Beteiligter. Wenn dem Gegenüber im Gespräch aber das Interesse fehlt, wird es ihm schwer fallen dieses weiter zu führen und er es wahrscheinlich schnell beenden. Außerdem wird er dann vielleicht in der Zukunft auch kein Gespräch mehr führen wollen und es kann keine Freundschaft entstehen. Genau so ist es auch mit den Meinungen. Wenn unterschiedliche Meinungen aufeinander treffen und im Alltag relevant sind und daher häufig aufkommen, kommt es wahrscheinlich regelmäßig zu Debatten bis hin zum Streit, wenn jeder auf seiner Position beharrt. Eine Freundschaft in der nur gestritten wird, scheint von außen betrachtet auch keine gute Freundschaft zu sein.\newline
Ein Roboter sollte also möglichst die Interessen der Nutzer teilen, wenn eine Freundschaft zwischen ihnen entstehen soll. Bei der Meinung ist es, ähnlich wie bei dem Thema der politischen Einstellung, sehr schwierig zu sagen wie weit man Meinungen und Ansichten in eine Maschine einprogrammieren sollte.\newline
Cicero findet allerdings noch eine andere Begründung wieso Wesen der eigenen Kulturgruppe uns näher stehen als Fremde. \glqq Darum sind uns unsere Mitbürger wichtiger als Ausländer, Verwandte wichtiger als Fremde; denn mit jenen hat die Natur selbst Freundschaft gestiftet, aber sie hat nicht genug Festigkeit. Darin nämlich hat die Freundschaft einen Vorzug vor der Verwandtschaft, dass das Wohlwollen aus der Verwandtschaft hinweggenommen werden kann, aus der Freundschaft aber nicht\grqq(Cicero, s.54,). Man ist mit den erwähnten Leuten also schon von Natur aus verbunden und diese Verbundenheit kann nicht verloren gehen, egal wie wenig man sich um diese Personen kümmert. Zu einer Freundschaft gehört aber mehr als Verbundenheit. Cicero nennt es hier Wohlwollen und meint damit, wie eine Person sich um eine andere sorgt und um sie kümmert. Eine Verwandtschaft schafft also schon eine Verbundenheit, die sicherlich eine gute Grundlage sein kann, aber ohne Wohlwollen kann trotzdem keine Freundschaft entstehen. Menschen untereinander haben also schon die Grundlage, dass sie von derselben Art sind, wohingegen Roboter und Menschen sehr unterschiedlich sind und die Unterschiedlichkeit die erste Hürde für eine Freundschaft darstellt.

\subsection{Freundschaft unter den Guten}
Im letzten Abschnitt wurde erwähnt, dass es keine Freundschaft ohne Wohlwollen geben kann. Damit sich zwei Menschen so gut verstehen, dass sie sich als Freunde sehen, müssen beide für den jeweils anderen das Beste wollen. Um für eine andere Person aber das Beste zu wollen, muss eine gewisse Gutartigkeit im Menschen bestehen. Denn Menschen, die nicht gut sind, wollen auch nur das Beste für sich selbst und interessieren sich weniger für die Gefühle anderer. Cicero definiert die Guten als \glqq Die Männer, die in ihrem Benehmen im ganzen Leben Treue, Rechtschaffenheit, Billigkeit und Edelmuth bewähren und frei von aller Leidenschaft, Zügellosigkeit oder Frechheit sind\grqq(Cicero, s.54). Die Treueeigenschaft ist sicherlich ein wichtiger Aspekt. Denn wenn ein Freund treu ist, kann man sich auf ihn Verlassen und die Freundschaft hat länger Bestand. Bei einem weniger treuen Freund, welcher vielleicht dazu neigt, seine Freunde häufiger zu wechseln, existiert weniger Vertrauen und auch durch die begrenzte Zeit kann keine tiefere Freundschaft entstehen. Denn \glqq zur Bildung solcher Herzensbünde [bedarf es] der Zeit und der Gewohnheit \grqq(Aristoteles, Buch 8, Kapitel 4).\newline
Rechtschaffenheit, Billigkeit und Edelmut beschreiben eine Person, die immer rechtmäßig handelt und so kaum mit falschen Entscheidungen oder ungerechten Taten polarisieren könnte. Eine solche Person ist sicherlich eine liebenswerte Person, da sie einem keine Gründe gibt, etwas gegen sie sagen zu können. Allerdings ist dies auch eine sehr einseitig und vielleicht daher auch langweilige Person. Frei von aller Leidenschaft, Zügellosigkeit oder Frechheit sind schließlich auch Begriffe, die eine Person beschreiben, die den Eindruck erweckt nicht besonders interessant zu sein. Die Person scheint kantenlos und ohne nennenswerte persönliche Eigenschaften. Auf eine Art scheint diese Person auch wie ein Roboter zu sein, der sich an Regeln hält und immer nur Gutes tut, weil er so programmiert wurde. Eine Freundschaft zu solch einer einseitigen Person, scheint allerdings den Spaß am Leben nicht sonderlich zu fördern. Mit Sicherheit können diese Menschen oder Roboter gute Freunde sein, allerdings sind es oft auch die Dinge, wie eine gewisse Frechheit oder ein gemeinsames Verstoßen gegen Gesetze oder Normen, die eine Freundschaft umso fester werden lässt. Und gerade eine gemeinsame Leidenschaft für manche Dinge verbindet Menschen ganz besonders.\newline
Im Gegensatz zu Cicero würde ich für einen Roboter, der zur Freundschaft entwickelt wird also nicht nur gute, gerechte Eigenschaften implementieren, sondern auch kleine Ecken und Kanten in der Persönlichkeit, welche ihn besonders machen.\newline
Nach der Definition von Cicero bin ich nicht der Meinung, dass Freundschaft nur unter den Guten entstehen kann. Allerdings beschreibt Aristoteles eine Freundschaft zu älteren und mürrischen Gemütern (vgl. Aristoteles, Buch 8, Kapitel 6) als wenig erstrebenswert, da es keinen Spaß macht, sich mit solchen Menschen auseinander zu setzten und es so wohl kaum jemand lange bei ihnen aushalten könnte. Unter diesem Gesichtspunkt wird deutlicher was mit Freundschaft unter den Guten gemeint ist und dem kann ich auch eher zustimmen. Mit einer griesgrämigen Person befreundet zu sein, bringt einem im Leben wenig Spaß, sondern eher mehr Ärger und schlechte Gedanken.\newline 
Allerdings bieten gerade diese Menschen den Robotern eine besondere Chance. Eine Maschine stört sich nicht daran, wenn sie schlecht behandelt wird. Sie bleibt freundlich und hält den Kontakt. So könnte also auch für mürrische Menschen ein Roboter der Freund sein, den sie unter den Menschen nicht finden können.\newline
Da einem Roboter also egal ist, ob das Gegenüber gut oder schlecht ist, ist diese Seite der Freundschaft einfacher. Allerdings müsste ja für den Menschen trotzdem noch der Roboter gut sein, um mit ihm befreundet sein zu wollen. Dazu stellt sich die Frage, ob eine Maschine überhaupt gut sein kann. Diesen Aspekt kann man allerdings leicht vernachlässigen mit dem Argument  \glqq Liebenswert ist was als gut erscheint \grqq (Aristoteles, Buch 8, Kapitel 2). Der Roboter muss also nicht per se gut sein, sondern nur auf uns so wirken als würde er es sein. Und das ist schließlich das Spezialgebiet eines Roboters, uns vorzumachen, dass er etwas denkt oder fühlt.

\section{Anwendungsgebiete}
In den vorherigen Kapiteln wurde nun festgestellt, dass eine Freundschaft zu einem Roboter nicht vollständig mit der Freundschaft unter Menschen zu vergleichen ist. Eine Mensch-Roboter-Freundschaft ist also wohl eher eine eigene Art von Freundschaft, wenn man es denn überhaupt so nennen möchte. Doch konnte auch gezeigt werden, dass sich Menschen genug auf eine Maschine einlassen kann, um zu einem persönlichen Verhältnis mit ihr zu gelangen. Um zu verstehen, wofür so ein Verhältnis nützlich sein kann, werden wir uns im folgenden Kapitel mit den Anwendungsbeispielen von Pflegerobotern im Altersheim sowie Babysitter Roboter für Kinder beschäftigen.

\subsection{Pflegeroboter}
Gerade bei älteren Leuten im Altersheim fehlen oft die sozialen Kontakte. Viele Verwandte und Freunde sind vielleicht schon gestorben und die Familie hat oft wenig Zeit, um die Senioren zu besuchen. Laut Aristoteles\glqq [erwächst] den Greisen [aus der Freundschaft] die wünschenswerte Pflege und Ersatz für das, was ihre Schwäche selbst nicht mehr vermag \grqq (Aristoteles, Buch 8, Kapitel 1). Doch wenn es diese Freunde nicht mehr gibt, muss das Pflegepersonal diese Aufgabe übernehmen. Doch da gerade in der heutigen Zeit auch dieses immer knapper wird und die wenigen Pflegekräfte immer weniger Zeit haben, um sich um die soziale Gesundheit der Patienten zu kümmern, wird versucht diese Aufgabe auf Roboter zu übertragen. Für genau diesen Zweck wurde die Roboterrobbe Paro entwickelt. Sie reagiert mit Bewegungen und Geräuschen, wenn sie gestreichelt wird und wird in einigen Altersheimen in Deutschland benutzt.\newline
In einigen Studien wurde herausgefunden, dass solche Roboter bei älteren Menschen das Gefühl der Einsamkeit lindern können und so einen positiven Effekt auf ihr Leben haben können (vgl. Oliver Bendel, s.79). Allerdings wurde auch festgestellt, dass die Zuwendung von Robotern keiner echten menschlichen Zuwendung entsprechen kann. Sie ist also bestenfalls eine gute Hilfestellung, aber in keiner Art und Weise als Ersatz zu sehen.\newline
Allerdings wurden diese Studien nie mit einem Roboter getestet, der dem Menschen verwechselbar ähnelt. Dieser Effekt könnte also auch darauf zurückzuführen sein, dass der Pflegeroboter einfach nicht authentisch genug war, um alle Aspekte, die ein Mensch in sozialen Kontakten sucht, widerzuspiegeln. Es ist also nicht sicher zu sagen, ob ein Roboter denselben sozialen Einfluss haben kann wie ein Mensch. Sicher ist nur, dass mit den Robotern die aktuell auf dem Markt sind, kein Verhältnis vergleichbar zum Menschen existieren kann.\newline
Überträgt man dies jetzt auf die Freundschaft, wird klar, dass ein Roboter nach aktuellem Entwicklungsstand vermutlich auch nicht den gleichen Einfluss auf einen Menschen haben kann wie ein menschlicher Freund. Es bleibt aber zu klären, ob zukünftige Entwicklungen diesem Verhältnis näher kommen können.

\subsection{Babysitter Roboter}
Besonders bei Kindern zeigt sich immer wieder wie schnell Freundschaften geschlossen werden, ohne über Bedingungen nachzudenken. Sie wählen ihre Freunde nicht nach dem Aussehen oder der Herkunft. Es kommt nur darauf an, dass man sich versteht und miteinander spielen kann. So ein Roboter könnte für ein Kind also schnell ein Gefährte sein, dem es vertraut und auf den es sich einlässt und ihn als Freund bezeichnet. Für Aristoteles liegt der Wert der Freundschaft für junge Menschen in der \glqq Bewahrung vor Fehltritten\grqq (Aristoteles, Buch 8, Kapitel 1). Dabei könnte ein Roboter auf das Kind aufpassen, ihm Dinge beibringen und es davon abhalten Fehler zu machen. Für Eltern könnte so ein Roboter eine Art Babysitter sein, um auf das Kind aufzupassen und es zum Beispiel frühzeitig ins Bett zu bringen, ihm Essen zu machen oder mit ihm zu spielen. In vielen dieser Aufgaben würde dem Roboter eher eine Position ähnlich zu einem Elternteil zukommen, aber besonders, wenn es darum geht sich mit dem Kind zu beschäftigen und mit ihm zu spielen, könnte mit dem Roboter ein freundschaftliches Verhältnis entstehen. Eine solche Roboter-Kind-Beziehung sollte allerdings mit Vorsicht betrachtet werden. Eine Freundschaft zu Robotern ist,  wie bereits erwähnt, nicht zu vergleichen mit der zu einem Menschen. Wenn aber ein Kind eine Freundschaft mit einem Roboter aufbaut und durch ihn lernt was Freundschaft bedeutet, könnte es zu dem Schluss gelangen, dass Freundschaft auch zu Menschen auf die gleiche Art und Weise funktioniert. Das bedeutet, dass es von einem Menschen nicht erwartet, dass man auf ihn Rücksicht nehmen muss, da es vom Roboter gewohnt ist, dass dieser nie schlechte Laune hat, sich nie angegriffen fühlt und selbst auch nie schlecht handeln würde. Das setzt natürlich voraus, dass ein Roboter auf diese Art und Weise programmiert ist. Sicherlich könnte man einen Roboter auch so programmieren, dass er wie ein Mensch manchmal schlechte Laune simuliert, beleidigt sein kann oder auch gemeine Äußerungen von sich geben könnte. Da das aber zu einer eher unnützen Maschine führen würde, wird dies wahrscheinlich nicht so programmiert werden. Allerdings könnte man sich eine Einstellungsmöglichkeit vorstellen, die genau das simuliert. Auf diese Weise könnte der Roboter in seiner wichtigen Position als Aufpasser in den Verantwortungsmodus gehen und in der Position als Freund auch negativ zu sehende menschliche Eigenschaften erfüllen, wie zum Beispiel Tollpatschigkeit, Albernheit oder Frechheit, um dem Kind beizubringen auch damit umzugehen.\newline
Es besteht allerdings außerdem die Gefahr, dass Eltern den Robotern viele ihrer Aufgaben überlassen und selbst immer weniger mit dem Kind interagieren. Sherry Turkle bezeichnet diese Situation als robotisches Moment und sieht es als sehr gefährlich an, dass die Interaktionen mit Menschen gerade in so wichtigen Lebensphasen wie Kindheit vernachlässigt werden (vgl. Catherine de Lange 2013). Kinder müssen in ihren ersten Lebensjahren sehr vieles lernen wie zum Beispiel auch das Zeigen und Interpretieren von Emotionen. Dazu ist es natürlich wichtig, diese Emotionen bei anderen Menschen zu sehen und mit ihnen über ihre eigenen Emotionen zu kommunizieren. Viele Kritiker fürchten die Gefahr, dass durch zu viel Umgang mit Robotern das Kind keine menschlichen Emotionen mehr lernen kann. Diese Ansicht geht allerdings davon aus, dass Roboter menschliche Emotionen nicht authentisch darstellen können. Aber auch das ist ein Bereich, der in der Zukunft mit Sicherheit immer besser gelingen kann. Tatsächlich gibt es ein EU-Forschungsprojekt DE-ENIGMA, welches mit Hilfe eines Roboters autistischen Kindern beibringen soll, Emotionen zu erkennen und zu erzeugen. Die Kommunikation zu Robotern soll für diese Kinder weniger einschüchternd wirken als zu Menschen und hilft ihnen daher mehr als ein echter Mensch. Es ist also nicht auszuschließen, dass eine solche Sorge unbegründet sein kann, sofern der Roboter nur authentisch genug designt ist und sich ebenso verhält.

%
% ---- Schlusswort ----
%
\section{Fazit}
Eine Freundschaft zwischen Menschen und Robotern scheint an sich nicht unmöglich zu sein. Jedoch wird sie sich trotzdem um einiges von der Freundschaft zwischen Menschen unterscheiden. Jegliche Aspekte hängen allerdings immer von der Bauweise und der Programmierung solcher Roboter ab. \newline
Bei der Bauweise kommt es nicht alleine darauf an, wie menschlich ein Roboter aussieht. Es ist viel wichtiger, dass die Bauweise und Verhaltensweise kohärent sind, sodass die Erwartungen, die mit dem Aussehen einhergehen erfüllt werden können. Wenn diese beiden Aspekte auseinander gehen, verschwindet der Roboter im Uncanny Valley und wirkt auf uns gruselig und nicht vertrauenswürdig.\newline
Auf der anderen Seite muss aber auch das Verhalten des Roboters authentisch sein. Er muss uns glaubwürdig vorspielen Emotionen zu haben, unser Interesse an einem Gespräch mit ihm wecken und sich an frühere Geschehnisse erinnern, um auch über solche Erfahrungen reden zu können.
Schwierige Themen sind allerdings seine eigene Meinung oder politische Einstellung, da fraglich ist, wie diese Meinung auszusehen hätte und ob diese nicht zu Konflikten mit dem Menschen führen könnten, wodurch eine Freundschaft nicht mehr funktionieren würde.\newline
Bei verschiedenen Definitionen von Freundschaft steht die gegenseitige Zuneigung oft im Vordergrund. Bei Robotern ist dies allerdings schwierig, da sie keine menschlichen Emotionen haben, sondern uns diese nur vorspielen. Es scheint aber dadurch für uns, als würde diese Zuneigung existieren und führt daher zu einer scheinbaren gegenseitigen Zuneigung und damit zu einer scheinbaren Freundschaft. \newline
Doch Aristoteles und Cicero scheinen besonderen Wert darauf zu legen, dass Freundschaft nur unter Gleichen bestehen kann. Auch eine Studie der Universität Bielefeld zeigte, dass Personen einem Roboter, der scheinbar der eigenen kulturellen Gruppe angehört, mehr vermenschlichen und als vertrauenswürdiger einschätzen, als Roboter einer Fremdgruppe. Außerdem sind gleiche Interessen und Meinungen hilfreich für fortlaufende Gesprächsstoffe in einer längeren Freundschaft. Allerdings ist es hier, wie bereits erwähnt, schwierig dies richtig zu implementieren und moralisch zu vertreten.\newline
Für Aristoteles kann Freundschaft außerdem nur unter den Guten entstehen. Eine Freundschaft zu griesgrämigen Menschen wäre schlichtweg sehr anstrengend und daher eher unwahrscheinlich zu wollen. Allerdings ist ein guter Mensch nach der Definition von Cicero ein kantenloser, netter Mensch, der sich nie von seiner Linie entfernen würde. Freundschaften machen für mich allerdings mehr aus als nur Nettigkeiten untereinander. Daher würde ich diesem Argument nicht zustimmen. Auch ein Roboter sollte daher Eigenschaften haben, wie ein gewisses Maß an Frechheit, welches ihn besonders und dadurch liebenswert machen könnte.\newline
Ein Roboterfreund wäre besonders nützlich in der Pflege und bei der Kinderbetreuung. Bei Senioren könnte er Einsamkeit vermindern und so die Pflegekräfte unterstützen. Studien zeigen allerdings, dass diese Roboter keine menschliche Zuwendung vollständig ersetzten können. Allerdings ist es fraglich inwieweit diese Studien davon abhängen wie weit Roboter entwickelt sind und ob menschlichere Roboter diesen Effekt irgendwann verbessern könnten.\newline
Bei der Betreuung von Kindern besteht die Gefahr, dass Kinder nicht unterscheiden können, wie sie sich gegenüber einem Roboter und einem Menschen verhalten müssen. Solange Roboter sich nicht so weit menschlich verhalten, dass es nicht mehr zu unterscheiden ist, birgt dies Gefahren für den Umgang von Menschen untereinander.\newline
Aber auch das Lernen von Emotionen könnte beeinträchtigt werden, solange Roboter die Emotionen anders herüberbringen als Menschen. Auch hier ist wieder zu sehen, dass es sehr darauf ankommt, wie weit und menschenähnlich ein Roboter entwickelt ist. Je menschenähnlicher er ist, desto weniger Gefahren birgt er für die Entwicklung von Kindern.\newline
Abschließend würde ich sagen, dass Menschen und Roboter Freunde sein können. Allerdings ist diese Art der Freundschaft nicht mit der zu einem Menschen zu vergleichen und birgt außerdem Gefahren in dem Umgang der Menschen untereinander und sollte daher mit Vorsicht betrachtet werden.

\newpage
%
% ---- Literatur ----
%
\begin{thebibliography}{5}
%

\bibitem{Aristoteles}
Aristoteles, übersetzt von Eugen Rolfes (1911): {\sl Nikomachische Ethik}. 

\bibitem{Cicero}
Cicero (2017) {\sl Philisophische Schriften Cicero}. Altenmünster, Jazzybee.

\bibitem{Guardian}
Catherine de Lange (2013): Sherry Turkle: 'We're losing the raw, human part of being with each other'. {\sl The Guardian}

\bibitem{Soziale Roboter}
Jens Koolway (2018): {\sl Die soziale Welt der Roboter}. Bielefeld: Transcript Verlag.

\bibitem{Dissertation}
Birte Schiffhauer (2015): {\sl Determinanten von Anthropomorphismus und ihre Bedeutung für Dehumanisierung. Zuschreibung und Absprechen von Menschlichkeit gegenüber Menschen und nicht-menschlichen Entitäten.} Bielefeld: Universität Bielefeld.

\bibitem{Pflegeroboter}
Oliver Bendel (Hrsg.) (2018): {\sl Pflegeroboter}. Wiesbaden: SpringerGabler.

\bibitem{Scheutz}
Matthias Scheutz (2009): {\sl The Inherent Danger of Unidirectional Emotional Bonds between Humans and Social Robots}. Medford/Somerville: Tufts University.

\bibitem{Alone together}
Sherry Turkle (2017): {\sl Alone Together. Why We Expect More from Technology and Less from Each Other}. New York: Basic Books.

\bibitem{Social categorization}
Eyssel und Kuchenbrandt (2011): {\sl Social categorization of social robots: Anthropomorphism as a function of robot group membership}. In British Journal of Social Psychology (2012), 51. 

\bibitem{duden}
Duden (Hrsg.): {\sl Freundschaft}. \url{ https://www.duden.de/suchen/dudenonline/freundschaft/ } Zuletzt eingesehen am 19.07.2019.

\end{thebibliography}

\end{document}
